\documentclass[12pt]{article}
\usepackage[legalpaper, margin=0.5in]{geometry}

\author{Tim Pasquel}

\title{CSC354 Software Engineering}

\begin{document}

\maketitle

\titlepage

\section{Jan Notes:}

\subsection{Jan 24 2024}

\begin{itemize}
		  \item \textbf{Steps:} Define solution (Find the goal with the client), design, code  
		  \item Prototype the hardest thing 1st so you can fail fast 
		  \item \textbf{Optional Books:} \it{Peopleware: Productive Teams and Projects, 3rd Eidtion, Tom Demarco \& Tim Lister, Addison-Wesley}
\end{itemize}

\subsection{Jan 26 2024}

\textbf{Project Ideas:}

More info can be found on Openai

\begin{enumerate}
		  \item Community Safety Network App
		  \item Personal Finance Coach (Similar to the Wii fit toaster) 
		  \item Waste Reduction Tracker
		  \item Plant Care and Identification 
		  \item DIY Home Improvement Assistant 
		  \item Speed test, but with system applications as variables 
		  \item Plant monitor with raspberry pi 
\end{enumerate}

\textbf{Good Idea:} Plant Monitor that uses a raspberry pi with a water sensor
\begin{itemize}
		  \item Use a pi zero 
		  \item use a pi water sensor found here: https://newbiely.com/tutorials/raspberry-pi/raspberry-pi-water-sensor
		  \item Create a python script that can read the water level data, help can be found on the website above 
		  \item Craete either a app or a webapp that can recive this information from the pi
		  \item Have the user create an account and then put in plant data
		  \item App will then read the plant needs and give the user notifications if the water level is too low
		  \item \textbf{Opt:} Have an automatic watering system if the water level gets too low
		  \item \textbf{Opt:} Figure out a way to have multiple of these devices running on one raspberry pi
		  \item Have a library on the app with common water levels for plants 
		  \item \textbf{Opt:} Have a chat board in which users can dicuss the best water options for plants 
\end{itemize}

\subsection{Jan 29 2024}

Project Pitches

\begin{itemize}
		  \item \textbf{Edutainment:} Game, 5 People, typing game to teach kids about jobs 
		  \item \textbf{ghest:} Motion device that pairs with bluetooth devices (Play, audio, video, messaging)
		  \item \textbf{PlantPro:} Me :). Additions: How will I power it, more sensors
		  \item \textbf{Have a Hobby:} App. that helps people find hobbies that they enjoy, reddit like, for 
					 the local location (set a range).
		  \item \textbf{Camera Detection:} Pi device that Detects IR waves and display it to the user, similar
					 to a flipper zero
\end{itemize}

\subsection{Jan 31 2024}

\begin{itemize}
		  \item Road Map: Scope, Planning, Requirements, Architecture, Prototype, Test Plan, Design, Sprints, Delivery
		  \item \textbf{Scope:} Determine whats \underline{inside} and \underline{outside} the system. 
					 \textbf{Artifacts:} Context diagram, user stories, use cases
		  \item \textbf{Planning:} Whats skills does the team \underline{have/need}. Who is responsible for what. 
					 \textbf{Artifacts:} Soft. Dev. Plan, PERT/Gantt Chart
		  \item \textbf{Requirements:} What are the \underline{functional} and \underline{non functional} 
					 requirements. User workflow? \textbf{Artifacts:} Software requirement specification
		  \item \textbf{Architecture:} Based on requirements, \underline{what servers} and \underline{hardware} are needed? 
					 Development environments? \textbf{Artifacts:} Architecture Diagram
		  \item \textbf{Planning pt2:} What is the \underline{highest risk} portion of the application? What do you understand
					 the least? \textbf{Artifacts:} Software Dev. Plan
		  \item \textbf{Prototype:} Design/build a working prototype. Research unknown items. Try different solutions. 
					 Create the prototype. \textbf{Artifacts:} Working prototype
		  \item \textbf{Test Plan:} Create system test plan based on \underline{user stories},
					 \underline{functional requirements}, \underline{non functional}
					 requirements. \textbf{Artifacts:} Test Plans
		  \item \textbf{Design:} Create a system level \underline{Class Diagram}. \textbf{Artifacts:} 
					 Detailed Design Specification
		  \item \textbf{Sprints:} Many sprints, each creates \underline{user functionality}. Each one is 2 weeks, goal is 4 total.
					 Sequence diagrams, code, test, deploy. \textbf{Artifact:} Detailed design specifaction
		  \item \textbf{Delivery:} Final user acceptance testing/report. How do you \underline{install} the system?
					 How do you \underline{use} the system? Demo day? \textbf{Artifacts:} Testing report, user manual, installation
					 guide.
\end{itemize}

\subsection{X}

\begin{itemize}
		  \item 
\end{itemize}

\end{document}
