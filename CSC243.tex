\documentclass[12pt]{article}
\usepackage[legalpaper, margin=0.5in]{geometry}

\author{Tim Pasquel}

\title{CSC243 Java Programing}

\begin{document}

\maketitle

\titlepage

\section{Jan Notes:}

\subsection*{Jan 23 2024}

N/A

\subsection*{Jan 25 2024}

N/A

\subsection*{Jan 30 2024}

How to create and run jar files:

\begin{enumerate}
		  \item Write code 
		  \item Compile: javac (filename).java
		  \item Execute: java -cp . (classpath)
		  \item Create Manifest.txt: Main-Class: (class) [\underline{Note:} Need to have a blank line 
					 at the end of the file]
		  \item Create jar file: jar cfmv (filename).jar Mainifest.txt (filename).class
		  \item Execute jar file: java -jar (filename).jar
\end{enumerate}

\begin{itemize}
		  \item Java is good because the jar files compiled can run on \textbf{ANY} device
					 as long as it has a java virtual machine
		  \item The Program:
					 \begin{itemize}
								\item State
								\item Processing
								\item \underline{Data} [CRUD: Create, Retrieve, Update, Delete]
					 \end{itemize}
		  \item Name of the java file \underline{must} be the name of the java file
\end{itemize}

\subsection*{Feb 1 2024}

\begin{itemize}
		  \item turnin243 <filename.jar>
\end{itemize}

\subsection*{Feb 6 2024}

N/A

\subsection*{Feb 8 2024}
   	
N/A

\subsection*{Feb 15 2024}

Sprints:

\begin{itemize}
		  \item Week 1 - Tuesday: Handout Program
		  \item Week 1 - Thursday: Submit Test Plan
		  \item Week 2 - Tuesday: Test Plan Assessment
		  \item Week 2 - Thursday: Submit Program, Submit Assessment, Progress Report 
\end{itemize}

\subsection*{Feb 20 2024}

Math functions

\begin{itemize}
		  \item Trigonometric methods (sin, cos, tan, acos, asin, atan, etc)
		  \item Exponent methods 
		  \item Rounding methods (ceiling, floor)
		  \item Min, max, abs, random methods
\end{itemize}

String and Char functions

\begin{itemize}
			\item Compare strings 
			\item Index strings
			\item Contains a string in a string
			\item Format strings
			\item Ignore case
			\item Concatinate strings
			\item To upper/lower case
\end{itemize}

Loops

\begin{itemize}
		  \item For, while, do while 
		  \item Use integers for controls, \underline{NOT} floats
		  \item Dont use semi colons at the end of a for loop
		  \item Use semi colon at the end of a do while
\end{itemize}

\subsection*{Mar 5 2024}

\begin{itemize}
		  \item \textbf{Object Oriented Programming:} The system accepts inputs from 
		  		the processing, the data, and the state. All of these external ``systems''
				are independent from one another.
		  \item \textbf{Classes:}
		  		\begin{itemize}
					\item Has a Constructor to define the object
					\item May have methods that act on that object
					\item May have variables that can be used in the object
					\item Should have set() functions
					\item Should have get() functions
					\item Should have mutators() functions
				\end{itemize}
		  \item \textbf{Class Diagrams:}
		  		\begin{itemize}
					\item Shows the class properties (Attributes and Associations)
				\end{itemize}
		  \item \textbf{Identifying Classes:} Start with the requirements, all nouns and info
		  		that your application needs to remember. Look at program plan
		  \item \underline{Data should be stored in one and only one place}
		  \item \textbf{Attributes:} <Visibility> <name>: <data type> <default value>
\end{itemize}

\subsection*{Mar 19 2024}

\begin{itemize}
		  \item \textbf{Relationships}
		  \begin{enumerate}
		  		\item \textbf{Association:} General binary relationship that describes
					an activity between two classes
				\item \textbf{Aggregation:} The child \underline{can} exsist without the parent 
				\item \textbf{Composition:} The child \underline{can not} exsist without the parent
		  \end{enumerate}
		  \item \textbf{Super Class:} The parent class in the relationship (Sets the common stuff)
		  \item \textbf{Sub Class:} The child class in the relationship (Has all of the stuff from the
				parent class, plus whatever is needed additionally for the unique sub class)
		  \item \textbf{Inheritance:} Child class takes resources (methods, variables, etc) from the parent class when created
		  \item \textbf{Polymorphism:} Child class takes resources (methods, variables, etc) from parent class when created and then 
		  		modifys it to meet its criteria
		  \item 
\end{itemize}

\subsection*{Feb X 2024}

\begin{itemize}
		  \item 
\end{itemize}

\end{document}

