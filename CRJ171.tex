\documentclass[12pt]{article}
\usepackage[legalpaper, margin=0.5in]{geometry}
\usepackage{titlesec}

\titleformat{\subsection}
{}
{}
{}
{} [\titlerule]

\author{Tim Pasquel}

\title{CRJ171 Asset Protection}

\begin{document}

\maketitle

\titlepage

\section*{Chapter 1}

\subsection*{Jan 22 2024}

N/A

\subsection*{Jan 24 2024}

\begin{itemize}
		  \item \textbf{Asset Protection:} Concerned with avoiding or mitigating loss. 
					 \begin{enumerate}
					 		\item An organizational function.
							\item Duty or a task (To report)
							\item Specific job (I.e Target [The Shopping Brand]
							\item Discipline or study (Loss Prevention Foundation [LPF],
									  ASIS International largest group of security personel)

					 \end{enumerate}
		  \item ``Gold standard'' for security management Certified Protection Professional
					(CPP)
		  \item Many companies swtiched from the term ``security'' to ``loss prevention''. Using
					 loss prevention can improve the image as people are expecting harsh terms with badges
		  \item \textbf{Loss Prevention:} Has its origin in the insurance industry. 
					 Risk can lead to loss, insurance companies focus on risk for their business.
		  \item \textbf{FEMA (opt.):} Federal Emergency Management Agency. Under the department of
					 homeland security. Designates floodplains and people in these plains are
					 at higher risk for flooding. 
\end{itemize}

\subsection*{Jan 26 2024}

N/A

\section*{Chapter 2}

\subsection*{Jan 29 2024}

\begin{itemize}
		  \item The language of Loss Prevention is the language of private security, which is the language of 
					 \underline{Business}, \textbf{NOT} Law Enforcement
		  \item \textbf{ROI:} Return on investment. How do you display ROI when you are a security firm?
		  \item \textbf{Security Services:}
					 \begin{enumerate}
								\item Proprietary (In House) 
								\item Contract [Varies State by State for liscensing]
					 \end{enumerate}
		  \item \textbf{Security Services pt2:}
					 \begin{enumerate}
								\item Armed 
								\item Unarmed
					 \end{enumerate}
		  \item Private Investigators need to be liscenced
		  \item \textbf{Security:} 
					 \begin{enumerate}
								\item Micro Level (Personal)
								\item Mezzo (Middle) Level (Organizational/Institutional)
								\item Macro Level (Communities/Federal)
					 \end{enumerate}
		  \item \textbf{Layers of Security:} 
					 \begin{enumerate}
								\item Deter
								\item Detect
								\item Delay
								\item Deny
								\item Asset
					 \end{enumerate}
		  \item Railroads and IoT (Internet of Things) have jurisdiction issues because these 
					 assets are accross many areas with different laws
\end{itemize}

\subsection*{Jan 31 2024}

\begin{itemize}
		  \item \textbf{The Retail Theft Act:} Defines what powers Law Enforcement have versus Loss Prevention on public
					 and private property
		  \item \textbf{Basic Hazards:} Natural, Technological, Human Caused. (*)D2L document
		  \item (*)Catalytic converter D2l document, why they are being stolen (Platnium)
		  \item \textbf{ROI:} Profitability ratio that determins income. This is hard for security/loss prevention. 
		  \item \underline{Annualized Loss Expected:} is used to figure ROI out, is determied by likelihood x impact. 
		  \item \textbf{Direct Losses:} Immidiate loss to a busniness such as the loss of cash in a safe
		  \item \textbf{In-Direct Losses:} The prolonged loss to a business such as the loss in sales due to a 
					 worse reputation due to a break in. 
\end{itemize}

\section*{Chapter 3}

\subsection*{Feb 5 2024}

\begin{itemize}
		  \item \textbf{Routine Activity Theory (\underline{Marcus Felson} \& \underline{Ronald V. Clarke (some)}):} 
					 Likely offender, Absence of Capable guardianship, suitable target
		  \item \textbf{Periodicals:} Peer reviewed journal. Universities and professors write these articles.
		  \item \textbf{\underline{Risk:}} = Likelihood of something occuring X vulnerability of location X 
					 consequence.
		  \item \textbf{Threat:} The likelihood of an attack (Something that is \underline{already} a hazard)
		  \item \textbf{Security Survey:} Address the organizations security concerns
		  \item \textbf{Risk Assessment:} Overall and systematic process of evaluting the effects of uncertainty
					 on achieving objectives. Likelihood X Vulnerability X Consequences
		  \item \textbf{Threat Assessment:} Book def. 
\end{itemize}

\subsection*{Feb 7 2024}

\underline{Ronald V. Clarke ideas:}

\begin{itemize}
		  \item \textbf{Criminology:} Nature, extent, causes/correlations of crime.
		  \item \textbf{Theory:} An explanation of what is going on, an idea, not nessessarily correct
		  \item \textbf{Environmental Criminology:} The study of changing the environment so ensure
					 a threat does not occur. (\underline{CEPTED}). \underline{ONLY} concerned with 
					 \underline{opportunity} aspect of an area.
		  \item He was based off of the \underline{classical school} of thought in which people have free will
					 and will do things based on oppertunity. 
		  \item \textbf{Rational Choice:} People/``criminals'' create a cost benefit analysis of whether or not they
					 should commit a crime. Therefore, loss prevention should increase the cost of execution,
					 and reduce the reward of the crime. (\underline{HeDonistic} Calculus).
\end{itemize}

\subsection*{Feb 9 2024}

\begin{itemize}
		  \item \textbf{Deterrence:} 
					 \begin{enumerate}
								\item Punishment has to be certain (Speeding ticket is written into law)
								\item Punishment has to be swift (Cop hands out ticket right then and there)
								\item (*) Punishment has to be severe (Speeding ticket hurts wallet, Razor fences can 
										  inflict pain)
					 \end{enumerate}
		  \item \textbf{CEPTED:} Crowe Zahm, have a better more secure environment will create more natual
					 survailence, people to feel better, good lighting, and sure territory borders. 
					 (Ex. the Arizona tribe in the side of a cliff, castles with a moat). Maintenance is a 
					 huge factor
		  \item \textbf{3M Glass:} Helps stop a shooter from entering a building as quickly
		  \item \textbf{Span of Control:} The number of employees that a person can effectively supervise
					 (Average 5 to 9 people at a time)
		  \item \textbf{Unity of Command:} Report to \underline{one} supervisor, not many at a time
		  \item \textbf{Democratic Style of Management:} For seasoned employees, everyone has input
		  \item \textbf{Autocratic Syle of Management} For novices, the manager has final say
		  \item \textbf{ASIS Internation:} Creates standards for the industry 
		  \item \textbf{ANSI Standard:} Works with the ASIS to create standards 
\end{itemize}

\section*{Chapter 4}

\subsection*{Feb 12 2024}

\begin{itemize}
		  \item (*)Outside reading on D2L: PA retail theft 18 PA. Cons. Stat.
		  \item \textbf{Common Law:} Also, known as case law, basis on law that evolved over time. Ex.
					 Murder, rape, arson, assult, etc.
		  \item \textbf{Case Law:} Judicial Precedent, Laws that have evolved from specific cases
					 Ex. Miranda Warnings (Miranda V. Arizona)
		  \item \textbf{Legislative Law:} Laws created by congress and state departments. Ex. Title 18 PA code
		  \item \textbf{Civil/Tort Law:} 
						\begin{itemize}
								  \item Private wrongs, civil or private wrongs, usually deals with money damages.
								  \item For compensitory damages (Health, days off, medical expenses),
											 for punitive actions (Usually negligence)
								  \item Defendent does not get a ``provided attorny'' so companies usually 
											 settle the case beforehand because lawyers are expensive. 
								  \item Both parities are allowed to appeal
						\end{itemize}
		  \item \textbf{Criminal Law:} Plaintiff (The Commonwealth [Crime against the state]) V. Defendant.
						\begin{itemize}
								  \item Petitioner V. Respondent. 
								  \item States usually do not appeal after losing a case, Defendants can appeal. 
								  \item Nobody is considered ``innocent'',
											 just not guilty \underline{``Beyond Resonable Doubt''}
						\end{itemize}
		  \item \textbf{Sources of Law Examples:} The Constituion, Tile 18 PA code, game laws, fishing laws,
					 boating laws, tax laws, inspection laws, motor vehicle code
\end{itemize}

\subsection*{Feb 14 2024}

\begin{itemize}
		  \item \textbf{Target Hardening:} Not a major CEPTED, creates prison like atmosphere. Barbed
					 wire fence is an example, make the target visibly more secure.
		  \item CEPTED should not interfere with the operational environment within scope. Ex. 
					 High schools shouldnt need metal detectors, airports should. Think about your
					 venue.
		  \item \underline{Loss Prevention Major Law Issues:}
					 \begin{itemize}
								\item \textbf{False Imprisonment:} Ex. Detaining someone by mistake
								\item \textbf{Assault:} Ex. More force than nessessary to deal with a person
								\item \textbf{Trespass:} Ex. Going onto someone elses property by accident (\$300 fine)
								\item \textbf{Negligence:} Ex. Inadiqite security
								\item \textbf{Wrongful Death:} Ex. Killing someone who did not need to die
					 \end{itemize}
		  \item \textbf{Cross Examination:} Hostile environment in court in which evidence is questioned from
					 expert witnesses, answer only the questions asked with (Yes, no, I dont recall). Dont
					 take things personally, stick to the facts, and remain calm.
		  \item \textbf{``Observe and Report'':} Post orders that state that all the guard does is 
					 watch and report what occurs on the shift. No physical repsonse to an event. Usually
					 for contractors. Limits liability on company
		  \item \textbf{Benchmarking:} What is the rest of the industry doing (Other motels,
					 stores, ASIS, companies, etc.)? Copy them.
		  \item \textbf{Armed/Unarmed:} Armed is much more expensive with large insurance costs as well
					 as training. Unarmed is usually ``observe and report'' and is just ``cheap security''
					 for due diligence and deterence
\end{itemize}

\subsection*{Feb 16 2024}

Chapter 4, 5

\begin{itemize}
		  \item Post 1950, there was an increase in \underline{new oppertunity} for crime due to
					suburbs, automobiles, women in the work force, highways, etc. 
		  \item Loss prevention mainly deal with minor courts/municiple courts (summary trials)
		  \item \textbf{Grading of Courts:}
		  		\begin{enumerate}
		  				\item Felony
						\item Misdemenor
						\item Summary
		  		\end{enumerate}
		  \item \underline{Infomration Below falls under PA Title 18 section 3929:}
		  \item Loss protection needs \underline{probable cause} to detain someone
		  \item Detention time for Loss prevention is determined by the company the person is working for. Ex.
		  		Target, Walmart, etc. (Usually 1 hour)
		  \item \textbf{Fingerprinting} is needed to see if the person detained has commited the crime before
		  \item Loss prevention team has a set of policys (similar to post orders) that should be followed
		  		to see how for they can go with arresting and chasing, etc.
\end{itemize}

\subsection*{Feb 21 2024}

Chapter 6

\begin{itemize}
		   \item Loss Prevention Program interacts with every department of an organization such as IT, Finance,
		  		Human Resources as well as external organizations such as Law Enforcement, news media, and hospitals
			\item Uniforms are set by the company/contract organization, uniforms psycologically influence
				people see D2L reading
			\item Employment law mainly falls under HR, loss prevention can be held accountable for things like
				``negligent hiring''
			\item \textbf{Polygraph:} ``Many to write''. Used for Federal occupations, L.E and the millitary.
				Private sector does not really do this due to hiring laws. Its not the machine that determines
				if it was a successful test, its the person asking the question. Mainly how the person acts under
				pressure.
			\item ``An ounce of prevention is worth a pound of cure'' - Benjamin Franklin 
				(To fix fires in Philidelphia)
		   \item Conduct background checks to help not have negligent hiring, negligent retention, harassement,
				hostile work environment
			\item Always look for social media as thats where people show their true colors
			\item Beware of \underline{deplomia mills} for fake accredidations
\end{itemize}

\subsection*{9}

\begin{itemize}
		  \item 
\end{itemize}

\end{document}

