\documentclass[12pt]{article}
\usepackage[legalpaper, margin=0.5in]{geometry}
\usepackage{titlesec}

\titleformat{\subsection}
{}
{}
{}
{} [\titlerule]

\author{Tim Pasquel}

\title{CRJ171 Asset Protection}

\begin{document}

\maketitle

\titlepage

\section*{Chapter 1}

\subsection*{Jan 22 2024}

N/A

\subsection*{Jan 24 2024}

\begin{itemize}
		  \item \textbf{Asset Protection:} Concerned with avoiding or mitigating loss. 
					 \begin{enumerate}
					 		\item An organizational function.
							\item Duty or a task (To report)
							\item Specific job (I.e Target [The Shopping Brand]
							\item Discipline or study (Loss Prevention Foundation [LPF],
									  ASIS International largest group of security personel)

					 \end{enumerate}
		  \item ``Gold standard'' for security management Certified Protection Professional
					(CPP)
		  \item Many companies swtiched from the term ``security'' to ``loss prevention''. Using
					 loss prevention can improve the image as people are expecting harsh terms with badges
		  \item \textbf{Loss Prevention:} Has its origin in the insurance industry. 
					 Risk can lead to loss, insurance companies focus on risk for their business.
		  \item \textbf{FEMA (opt.):} Federal Emergency Management Agency. Under the department of
					 homeland security. Designates floodplains and people in these plains are
					 at higher risk for flooding. 
\end{itemize}

\subsection*{Jan 26 2024}

N/A

\section*{Chapter 2}

\subsection*{Jan 29 2024}

\begin{itemize}
		  \item The language of Loss Prevention is the language of private security, which is the language of 
					 \underline{Business}, \textbf{NOT} Law Enforcement
		  \item \textbf{ROI:} Return on investment. How do you display ROI when you are a security firm?
		  \item \textbf{Security Services:}
					 \begin{enumerate}
								\item Proprietary (In House) 
								\item Contract [Varies State by State for liscensing]
					 \end{enumerate}
		  \item \textbf{Security Services pt2:}
					 \begin{enumerate}
								\item Armed 
								\item Unarmed
					 \end{enumerate}
		  \item Private Investigators need to be liscenced
		  \item \textbf{Security:} 
					 \begin{enumerate}
								\item Micro Level (Personal)
								\item Mezzo (Middle) Level (Organizational/Institutional)
								\item Macro Level (Communities/Federal)
					 \end{enumerate}
		  \item \textbf{Layers of Security:} 
					 \begin{enumerate}
								\item Deter
								\item Detect
								\item Delay
								\item Deny
								\item Asset
					 \end{enumerate}
		  \item Railroads and IoT (Internet of Things) have jurisdiction issues because these 
					 assets are accross many areas with different laws
\end{itemize}

\subsection*{Jan 31 2024}

\begin{itemize}
		  \item \textbf{The Retail Theft Act:} Defines what powers Law Enforcement have versus Loss Prevention on public
					 and private property
		  \item \textbf{Basic Hazards:} Natural, Technological, Human Caused. (*)D2L document
		  \item (*)Catalytic converter D2l document, why they are being stolen (Platnium)
		  \item \textbf{ROI:} Profitability ratio that determins income. This is hard for security/loss prevention. 
		  \item \underline{Annualized Loss Expected:} is used to figure ROI out, is determied by likelihood x impact. 
		  \item \textbf{Direct Losses:} Immidiate loss to a busniness such as the loss of cash in a safe
		  \item \textbf{In-Direct Losses:} The prolonged loss to a business such as the loss in sales due to a 
					 worse reputation due to a break in. 
\end{itemize}

\section*{Chapter 3}

\subsection*{Feb 5 2024}

\begin{itemize}
		  \item \textbf{Routine Activity Theory (\underline{Marcus Felson} \& \underline{Ronald V. Clarke (some)}):} 
					 Likely offender, Absence of Capable guardianship, suitable target
		  \item \textbf{Periodicals:} Peer reviewed journal. Universities and professors write these articles.
		  \item \textbf{\underline{Risk:}} = Likelihood of something occuring X vulnerability of location X 
					 consequence.
		  \item \textbf{Threat:} The likelihood of an attack (Something that is \underline{already} a hazard)
		  \item \textbf{Security Survey:} Address the organizations security concerns
		  \item \textbf{Risk Assessment:} Overall and systematic process of evaluting the effects of uncertainty
					 on achieving objectives. Likelihood X Vulnerability X Consequences
		  \item \textbf{Threat Assessment:} Book def. 
\end{itemize}

\subsection*{Feb 7 2024}

\underline{Ronald V. Clarke ideas:}

\begin{itemize}
		  \item \textbf{Criminology:} Nature, extent, causes/correlations of crime.
		  \item \textbf{Theory:} An explanation of what is going on, an idea, not nessessarily correct
		  \item \textbf{Environmental Criminology:} The study of changing the environment so ensure
					 a threat does not occur. (\underline{CEPTED}). \underline{ONLY} concerned with 
					 \underline{opportunity} aspect of an area.
		  \item He was based off of the \underline{classical school} of thought in which people have free will
					 and will do things based on oppertunity. 
		  \item \textbf{Rational Choice:} People/``criminals'' create a cost benefit analysis of whether or not they
					 should commit a crime. Therefore, loss prevention should increase the cost of execution,
					 and reduce the reward of the crime. (\underline{HeDonistic} Calculus).
\end{itemize}

\subsection*{Feb 9 2024}

\begin{itemize}
		  \item \textbf{Deterrence:} 
					 \begin{enumerate}
								\item Punishment has to be certain (Speeding ticket is written into law)
								\item Punishment has to be swift (Cop hands out ticket right then and there)
								\item (*) Punishment has to be severe (Speeding ticket hurts wallet, Razor fences can 
										  inflict pain)
					 \end{enumerate}
		  \item \textbf{CEPTED:} Crowe Zahm, have a better more secure environment will create more natual
					 survailence, people to feel better, good lighting, and sure territory borders. 
					 (Ex. the Arizona tribe in the side of a cliff, castles with a moat). Maintenance is a 
					 huge factor
		  \item \textbf{3M Glass:} Helps stop a shooter from entering a building as quickly
		  \item \textbf{Span of Control:} The number of employees that a person can effectively supervise
					 (Average 5 to 9 people at a time)
		  \item \textbf{Unity of Command:} Report to \underline{one} supervisor, not many at a time
		  \item \textbf{Democratic Style of Management:} For seasoned employees, everyone has input
		  \item \textbf{Autocratic Syle of Management} For novices, the manager has final say
		  \item \textbf{ASIS Internation:} Creates standards for the industry 
		  \item \textbf{ANSI Standard:} Works with the ASIS to create standards 
\end{itemize}

\section*{Chapter 4}

\subsection*{Feb 12 2024}

\begin{itemize}
		  \item (*)Outside reading on D2L: PA retail theft 18 PA. Cons. Stat.
		  \item \textbf{Common Law:} Also, known as case law, basis on law that evolved over time. Ex.
					 Murder, rape, arson, assult, etc.
		  \item \textbf{Case Law:} Judicial Precedent, Laws that have evolved from specific cases
					 Ex. Miranda Warnings (Miranda V. Arizona)
		  \item \textbf{Legislative Law:} Laws created by congress and state departments. Ex. Title 18 PA code
		  \item \textbf{Civil/Tort Law:} 
						\begin{itemize}
								  \item Private wrongs, civil or private wrongs, usually deals with money damages.
								  \item For compensitory damages (Health, days off, medical expenses),
											 for punitive actions (Usually negligence)
								  \item Defendent does not get a ``provided attorny'' so companies usually 
											 settle the case beforehand because lawyers are expensive. 
								  \item Both parities are allowed to appeal
						\end{itemize}
		  \item \textbf{Criminal Law:} Plaintiff (The Commonwealth [Crime against the state]) V. Defendant.
						\begin{itemize}
								  \item Petitioner V. Respondent. 
								  \item States usually do not appeal after losing a case, Defendants can appeal. 
								  \item Nobody is considered ``innocent'',
											 just not guilty \underline{``Beyond Resonable Doubt''}
						\end{itemize}
		  \item \textbf{Sources of Law Examples:} The Constituion, Tile 18 PA code, game laws, fishing laws,
					 boating laws, tax laws, inspection laws, motor vehicle code
\end{itemize}

\subsection*{Feb 14 2024}

\begin{itemize}
		  \item \textbf{Target Hardening:} Not a major CEPTED, creates prison like atmosphere. Barbed
					 wire fence is an example, make the target visibly more secure.
		  \item CEPTED should not interfere with the operational environment within scope. Ex. 
					 High schools shouldnt need metal detectors, airports should. Think about your
					 venue.
		  \item \underline{Loss Prevention Major Law Issues:}
					 \begin{itemize}
								\item \textbf{False Imprisonment:} Ex. Detaining someone by mistake
								\item \textbf{Assault:} Ex. More force than nessessary to deal with a person
								\item \textbf{Trespass:} Ex. Going onto someone elses property by accident (\$300 fine)
								\item \textbf{Negligence:} Ex. Inadiqite security
								\item \textbf{Wrongful Death:} Ex. Killing someone who did not need to die
					 \end{itemize}
		  \item \textbf{Depositions:} Part of the discovery process, used to obtain testimony from the witness
		  		for use later on in court. 
		  \item \textbf{Interaggatories:} Part of the discovery process, written questions for the opposing
		  			side of a case that must be answered.
		  \item \textbf{Cross Examination:} Hostile environment in court in which evidence is questioned from
					 expert witnesses, answer only the questions asked with (Yes, no, I dont recall). Dont
					 take things personally, stick to the facts, and remain calm.
		  \item \textbf{``Observe and Report'':} Post orders that state that all the guard does is 
					 watch and report what occurs on the shift. No physical repsonse to an event. Usually
					 for contractors. Limits liability on company
		  \item \textbf{Benchmarking:} What is the rest of the industry doing (Other motels,
					 stores, ASIS, companies, etc.)? Copy them.
		  \item \textbf{Armed/Unarmed:} Armed is much more expensive with large insurance costs as well
					 as training. Unarmed is usually ``observe and report'' and is just ``cheap security''
					 for due diligence and deterence
\end{itemize}

\subsection*{Feb 16 2024}

Chapter 4, 5

\begin{itemize}
		  \item Post 1950, there was an increase in \underline{new oppertunity} for crime due to
					suburbs, automobiles, women in the work force, highways, etc. 
		  \item Loss prevention mainly deal with minor courts/municiple courts (summary trials)
		  \item \textbf{Grading of Courts:}
		  		\begin{enumerate}
		  				\item Felony
						\item Misdemenor
						\item Summary
		  		\end{enumerate}
		  \item \underline{Infomration Below falls under PA Title 18 section 3929:}
		  \item Loss protection needs \underline{probable cause} to detain someone
		  \item Detention time for Loss prevention is determined by the company the person is working for. Ex.
		  		Target, Walmart, etc. (Usually 1 hour)
		  \item \textbf{Fingerprinting} is needed to see if the person detained has commited the crime before
		  \item Loss prevention team has a set of policys (similar to post orders) that should be followed
		  		to see how for they can go with arresting and chasing, etc.
\end{itemize}

\subsection*{Feb 21 2024}

Chapter 6

\begin{itemize}
		   \item Loss Prevention Program interacts with every department of an organization such as IT, Finance,
		  		Human Resources as well as external organizations such as Law Enforcement, news media, and hospitals
			\item Uniforms are set by the company/contract organization, uniforms psycologically influence
				people see D2L reading
			\item Employment law mainly falls under HR, loss prevention can be held accountable for things like
				``negligent hiring''
			\item \textbf{Polygraph:} ``Many to write''. Used for Federal occupations, L.E and the millitary.
				Private sector does not really do this due to hiring laws. Its not the machine that determines
				if it was a successful test, its the person asking the question. Mainly how the person acts under
				pressure.
			\item ``An ounce of prevention is worth a pound of cure'' - Benjamin Franklin 
				(To fix fires in Philidelphia)
		   \item Conduct background checks to help not have negligent hiring, negligent retention, harassement,
				hostile work environment
			\item Always look for social media as thats where people show their true colors
			\item Beware of \underline{deplomia mills} for fake accredidations
\end{itemize}

\subsection*{Feb 23 2024}

\begin{itemize}
		  \item \textbf{Polygraph Protection Act:} Makes it a misdeminor to use this as an employment
		 		prerequisite for the private industry. PA law. Only people with access to narcotics
			   or other dangerous drugs can require it. Very expense to polygraph	
		  \item \textbf{MMPI-2 (Minnesota Multiphased Personality Inventory):} Measures a persons personality.
		 		Sees if they are aggresive and such. Yes and no questions
		  \item \textbf{Spoils System:} When someone wins, its easy for people who worked for the person
		 		to work in whatever department or agency they want. Victory by patronage, not merit 
		  \item \textbf{Chester Aurthor:} Part of the spoils system, people hated how he was becoming president
		 		and passed the \underline{(*)[Reading] Civil Service Act} 
		  \item \textbf{Maslows Hierarchy of Needs (Abraham Maslow):} (*) basic needs that a person desires. 
		  		The lower levels
		  		of need to be meet before the higher ones. We are motivated to do them one tier at a time. 
				One can move up the pyramid as well as down if its not secure enough
		 		\begin{enumerate}
					\item Self Actualization (Reaching ones full potential)
					\item Esteem (Respect from others, self confidence)
					\item Belonging Needs (Friendships, family, groups)
					\item Security (Job, health, personal secuirty)
					\item Physilogical (Food, water, shelter)
			  	\end{enumerate}	
			\item Suble police uniforms for loss prevention should be blazers and sports coats
\end{itemize}

\subsection*{Feb 26 2024}

Chapter 7

\begin{itemize}
		  \item \underline{Transferable Skills:}
		  		\begin{itemize}
					\item Leadership 
					\item Public Speaking/Communication
					\item Sales
					\item Curiosity/Inquisitiveness
					\item Problem Solving
					\item Creativity
					\item Teamwork
					\item Digital Skills
					\item Grit/Perserverance
				\end{itemize}
\end{itemize}

\subsection*{Feb 28 2024}

\begin{itemize}
		  \item \textbf{Employee Theft:}  Theft by internal employees
		  \item \textbf{Pilferage:} Theft of small items at a time  
		  \item \textbf{Embezzlement:} Taking money that is entrusted to an employee from the company  
		  \item \underline{Research}
		 		\begin{itemize}
					\item \underline{Baxter} studied employee theft (Doctoral Disseration: PHD student's final
						major research project for their degree) 
					\item Questions (What are the items being stolen? Specific location of theft? 
						Who is being caught? Costs? Frequency of occurence? Reaction? Motivations?)
					\item The company you are researching needs to be completly on board with the research
					\item Data (Is plurarl). Where does it come from?. Ex. Employee records. case files, 
						confession statements (Interviewing-Wicklander ex), camera footage (lasts 30-60 days
						depending on industry), interview the loss prevention team
		  			\item Independent Variable: 
		  			\item Dependent Variable:
		  			\item Ex. What is the impact of CCTV (Independent Var) on preventing Auto theft (Dept. Var)?
			  	\end{itemize}	
		  \item \textbf{Shrink:} Amount of loss profit from an unknown internal cause. Currently 1.62\%, it 
		 		has been increasing over the years 
		  \item \textbf{Hypothosis:} Why is the theft/shrink occuring. Ex. Employee/manager turnover yields
		  		a higher theft rate. More managers means more employees which means more theft. 
		  \item \textbf{Social Strain:} Robert Merton - Argued crime occurs when there is a gap between
				goals (usually money) and legitimate means. Ex. selling drugs and theft makes more money than
				a regular job. 
		  \item \textbf{(WZ) Wicklander Zulawski Interview:} Loss prevention interview that figures out why an
				employee did what they did and why. A round about way of doing so, gives examples and 
				tries to gain trust with the person being interviewed. In class video. (*)A non confrontational
			   technique.	
		  \item \textbf{Nonprobability Sample:} Sample that is based on the researchers opinion. They decide
				who to study. Ex. CRJ students in college.  
		  \item \textbf{Secondary Data Analysis:} Data that has already been aquired by someone else
		  \item What can you give from your research that will beneifit the company? Why should they allow
				you to do research?  
		  \item \textbf{IRB:} Institutional Review Board - Used when you are dealing with interviewing/studying
				humans. Considered to be a service  
\end{itemize}

\subsection*{Mar 1 2024}

\begin{itemize}
		  \item \textbf{Sweethearting:} Giving people that you know such as family and friends free products  
		  \item \textbf{VToI:} Sees every item that is run up on the system  
		  \item \textbf{Report Writing in Asset Protection:} Who? What? When? Where? How? \underline{Why?}  
		  \item \textbf{\underline{CRAVED:}}
				\begin{itemize}
					\item \underline{C:} Concealable (Small objects fit in bags)
					\item \underline{R:} Removeable (Bikes and laptops are easily taken)
					\item \underline{A:} Available (Easy to find, things left on a car seat)
					\item \underline{V:} Valuable (Not nessessarly for resale)
					\item \underline{E:} Enjoyable (Liquor, tabacco)
					\item \underline{D:} Disposable (Ciggerettes)
					\item Razor Blades, Generators, Baby Formula are examples of these items
			 	\end{itemize}	
\end{itemize}

\subsection*{Mar 4 2024}

\begin{itemize}
		  \item \textbf{\underline{``The Four D's''}}
				\begin{enumerate}
			 			\item Deter (Discourage people, signs)	
			 			\item	Detect (Sensors, cameras)
			 			\item	Delay (Fencing, barriers)
			 			\item	Deny (Rising barriers, guards)
			 			\item	Asset
				\end{enumerate}  
			\item \textbf{CCTV Methods:}
				\begin{itemize}
						\item Patrol (Someone watching cameras physcially)
						\item Response (Cameras that are always on and are used forensically)
				\end{itemize}
			\item \textbf{CCTV AI:} Line crossing (Alert is generated when someone crosses a specific point), 
				intrusion detection (Object enters forbidin area), counting (Counting object crossing a line), 
				loitering (When an object stays in an area for a specific amount of time), damage resistent (if 
				someone tampers or obstructs the camera in anway an alert is generated)
			\item \textbf{People Analyitics (CCTV):}
				\begin{itemize}
						\item Person of Interest (Try to find a specific person)
						\item Person history (See who has been in and out of facility)
						\item Trends (Quantify foot traffic at certain location)
				\end{itemize}
			\item \textbf{Pixels per Foot:} PPF = Horizontal pixels / HFoV width. 3840p/ 12ft = 320 PPF
			\item \textbf{Pixels per Foot:} ``Picture Element'', smallest unit in digital display
			\item \textbf{MP:} Mega Pixel, 1 million pixels
			\item \textbf{Head Angle:} For optimal performance, person of interests should be captured head on face
				to face, not over head or behind the head, straight on.
			\item \textbf{ALPR:} License plate reader
			\item Dont place your cameras facing the sun or covered by foliage
			\item 360 degree cameras good for interior
			\item 180 degree for exterier use
			\item \underline{Casinos} spend millions of dollars on camera systems
			\item \textbf{Meta Analysis:} A systematic study of a study to insure integrity
			\item London has the highest count of security cameras
			\item Use a layered approach to security system, dont just use the cameras, do security survey to
				find out what the best method is when layering
			\item IP based cameras sometimes have \underline{lag} in which the footage is outdated to what is
				actually occuring. Not good for active shooters when the camera is 30 seconds behind

\end{itemize}

\subsection*{Midterm Notes}

\begin{itemize}
		  \item \textbf{Retail Theft Act}  
		  \item \textbf{Crow and Zahm}  
		  \item \textbf{Civil Law}  
		  \item \textbf{Psycological Influence of Police Uniform}  
		  \item \textbf{Span of Control/Unity of Command}  
		  \item \textbf{Peer Review}  
		  \item \textbf{Ideal span of control: 1 Supervisor, 10 employees}  
		  \item \textbf{Level of proof needed for both criminal and civil law cases}  
\end{itemize}

\subsection*{Mar 8 2024}

Chapter 8

\begin{itemize}
		  \item \textbf{Glazing/Glass:} 
		  \item \textbf{Film:}  
		  \item \textbf{Doors:}  
		  \item \textbf{Locks:}  
		  \item \textbf{Alarms:}  
		  \item \textbf{Cameras:}  
		  \item \textbf{Lighting:}  
		  \item \textbf{Vehicle Controls:}  
		  \item \textbf{Security Personnel:}  
		  \item \textbf{Policies:}  
\end{itemize}

\subsection*{Mar 18 2024}

\begin{itemize}
		  \item \textbf{Types of Glass:}  
				\begin{enumerate}
					\item \underline{Annealed:} Regular glass, can not take much force, 
						breaks into razor shards
				   \item \underline{Tempered:} Used in industry, breaks into balls of glass that are not 
						as sharp. 	
			 	\end{enumerate}
		  \item \textbf{Glazing:}  
				\begin{enumerate}
					\item Entry Level Security Film
					\item Glazing Security System
					\item 3M (Riot Glass) \underline{Most Popular}
					\item Acess Denial
					\item Balistic Glass
			 	\end{enumerate}
		  \item Do not strive for success, strive for absolutly no failure	
\end{itemize}

\subsection*{Mar 20 2024}

\begin{itemize}
		  \item \textbf{Transitional Spaces:} Areas that do not have \underline{capitable 
					guardianship} in between places that do. (Ex. Poorly lit areas)
		  \item \textbf{''Routine Percaution'':} Actions that individuals take for their own
		 			security. Ex. Having a firearm, install home security system	
		  \item \textbf{Diffusion of Benefits:} When you add one addition to security for one 
		  			issue, it could positivily impact other issues for security
		  \item \textbf{Displacement of Crime:} Making it so that crime has to occur in another
		  			region
		  \item \textbf{The Crime Operrtunity Structure:} Made up of Lifestyle/Routine Activity and the Physical Environment
		  \item \textbf{Post Orders:} Written procedures as to what officers are supposed to do. Should have phone numbers,
		  			Names and key personnel, open/close procedures, patrol routes, mission, what to do in event of...
		  \item \textbf{Types of Security:} Retail tends to be proprietary, construction tends to be contractual
		  \item To figure out if you need proprietary or contracual as well as armed vs unarmed, they should conduct a
		  			security survey
		  \item \textbf{Act 235:} The ability/License for a security officer to be armed. \underline{Varies by state}. 
		  			Enforced by the state police. Have to carry it with you while you are working
		  \item \textbf{Unarmed Loss Prevention:} There is no required training for loss prevention that is unarmed for PA, not
		  			all states are like this
		  \item \textbf{Private Detective Act of 1953:} Licences private detectives, invesigators, guard, and patrol agencies
\end{itemize}

\subsection*{Mar 22 2024}

\begin{itemize}
		
		  \item \textbf{Desk Receptionist Exercise:}
				\begin{itemize}
					\item I would feel better going to the blue desk for questions over the red one
			 	\end{itemize}	
		  \item \textbf{Hue:} Quality or characteristic by which one color is different from another
		  \item More than 10,000,000 colors in the world
		  \item \textbf{Saturation:} Strength, intensity, chromo of a color. Purity of a given color
		  \item \textbf{Lightness:} Differentiates a dark color from a light one. The pigmint
		  \item \textbf{Yellow:} A stimulating color, happier color, but also very stimulating
		  \item \textbf{Red:} Fire, heat, blood, anxiety, agitation, power, passion, triggers sensory items (smell, taste)
		  \item \textbf{Blue:} Loyalty, trust, peaceful, tranquility, depression, coldness, slows pulse rate, relaxes, opposite of red.
		  \item \textbf{Green:} Relaxing, calming, soothing, popular decorating color, ``green room'' relaxes stars before going on TV
		  \item \textbf{Orange:} Contentment, warmth, similar to red with less pronounced, stimulates learning
\end{itemize}

\subsection*{Mar 25 2024}

\begin{itemize}
		  \item \textbf{Investigation Questions:} Who, what, where, when, how? Tell a story, get as many
				details as possible (Time, snow, rain, location) 
		  \item \textbf{Proprietary Investigation:} In house investigation
		  \item \textbf{Contract Investigation:} Bought investigation, Pinkerton was the first PI (``The eye
				that never sleeps'')
		  \item (*) Every state is different for licensing for PIs   
		  \item Private Detective Act was made in the 1953 mainly for a position for retired State Troopers, has
		  		not been changed to this day
		  \item PIs can legally carry firearms under ACT 235
		  \item PIs can secretly record someone in places where privacy is not expected.
		  \item \textbf{One Party State:} Can record someone without the other person knowing  
		  \item \textbf{Two Party State:} Can only record someone with their their permission/with them knowing (PA
				is one of these states)	
\end{itemize}

\subsection*{Mar 27 2024}

\begin{itemize}
		  \item \textbf{Report Writing:} Report (not an opinion), use correct grammer so you know who you
				are reporting, 
		  \item \textbf{Best Answers for Cross Examination:} yes, no, I do not recall	
		  \item \textbf{For WZ questioning:} Open ended questions tended to get more detail out of the suspect
		  		over closed ended questions
\end{itemize}

\subsection*{Mar 8 2024}

\begin{itemize}
		  \item \textbf{CRAVED:}  
\end{itemize}

\subsection*{Mar 8 2024}

\begin{itemize}
		  \item \textbf{CRAVED:}  
\end{itemize}

\subsection*{Mar 8 2024}

\begin{itemize}
		  \item \textbf{CRAVED:}  
\end{itemize}

\end{document}

