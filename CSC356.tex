\documentclass[12pt]{article}
\usepackage[legalpaper, margin=0.5in]{geometry}
\usepackage{soul}

\author{Tim Pasquel}

\title{CSC356 Database Systems}

\begin{document}

\maketitle

\titlepage

\section{Chapter 1:}

\subsection*{Jan 23 2024}

\begin{itemize}
		  \item \textbf{Database:} Collection of \underline{related} stored data (Ex. KU database by Microsoft Access
		  \item Having multiple copies of the same data can lead to flawed, contradictory info. 
		  \item Org. can benefit by having their data integrated into a single database.
		  \item \textbf{Integrated Database:} Collection of related data that can be used simultaneously by many departments and users in an enterprise
		  \item \textbf{Metadata:} Data about data
		  \item \textbf{``System Catalog'':} How all of the tables are related to each other/a list of them all
		  \item \textbf{Database Administrator (DBA):} Responsible for creating and maintaining the database
		  \item \textbf{Database Management Systems (DBMS):} A s/w package that allows creating and maintaining databases. All access to the database is controlled by a DBMS
		  \item Apps that are written in java, c++, etc. go through DBMS and can then present data in the 
					 form each program expects 
		  \item \textbf{End Users:} People who use the data to perform their jobs
		  \item \textbf{(*)Advantages of Integrated Database:} Sharing of data, control of redundancy, data consistency, improved data standards (Who needs to know what in each table), better data security (Ensures all data is entered correctly), improved data integrity (Same as last adv.), balancing of conflicting requirements, faster development of new apps., better data accessibility, economy of scale (Easy to modify database), more control over concurrency (More control over events that happen at the same time), better backup and recovery procedures (One repository makes less redundencies)
\end{itemize}

\subsection*{Jan 25 2024}

(*) Notes book drawings/examples

\begin{itemize}
		  \item \textbf{Relational (Tables)  Database:} They use tables which are easy to understand as well as
					 easy to implement. 
\end{itemize}

\subsection*{Jan 30 2024}

(*) Notes book drawings/examples

\textbf{Historical Developments in Databases}

\begin{itemize}
		  \item \textbf{Sequential File Processing System:} Using sequential access devices such as: punched cards in the 1890 US census, punched paper tape introduced in 1940s, magnetic tape introduced about 1950-used in UNIVAC I
		  \item \textbf{Early Database Model:} S.F.P.S (1950s): Early DB MOdels (1960s)
					 \begin{itemize}
								\item Magnetic disk introduced 1950s which was a direct access device
								\item Languages like COBOL, PL/1 for commercial processing in 1960s
								\item \underline{Hierarchical Model:} Database developed 1960s 
								\item \underline{Network Model:} Database developed 1960s
								\item both were complex, required users to understand database structures and 
										  access paths to data (Where the folders were)
					 \end{itemize}
		  \item \textbf{Relational Model:} Proposed by E.F Codd 1970 
					 \begin{itemize}
								\item Strong mathematical theory/foundation/fuctions
								\item Table model
								\item Still popularly used
								\item Similar to Microsoft Excel (?)
					 \end{itemize}
		  \item \textbf{Semantic Models:}
					 \begin{itemize}
								\item Entity relationship model (E.R Model) - P.P Chen 1976. \underline{(*)Used today}
								\item Object oriented model (O.O Model) - 1990s 
								\item Object relational mode: Object oriented capabilities in a relational database
					 \end{itemize}
		  \item \textbf{Data Warehouse:}
					 \begin{itemize}
								\item 1990s, lots of data that can be used to do statistics or, predictions to help business. 
								\item Built to store large quantities of historical data
								\item Enables fast complex quries across all of the data
								\item Typically using OLAP (Online \underline{Analytical} Processing)
								\item \underline{Note:} DB is typically for OLTP (Online Transaction Processing)
					 \end{itemize}
		  \item \textbf{Internet Access:} to a network of databases (Web based data base, publicaly avaliable)
					 \begin{itemize}
								\item Ecommerce
								\item XML standard for data exchange
					 \end{itemize}
		  \item \textbf{Big Data:} The capturing, organizing, and analyzing of massive amounts of data
					 generated by large variety of sources
\end{itemize}

\section{Chapter 2}

\subsection*{Feb 1 2024}

(*) Notes book drawings/examples

\begin{itemize}
		  \item A resource is any asset of value to an organization, incurs cost to organizations
					 operational data
		  \item Recognition of data as \underline{corperate resource} is important objective in developing
					 integrated database environment
		  \item Database secures data with a \textbf{DBMS} which provides \underline{security, integrity,
					 and realiability}
		  \item \textbf{Data:} Bare facts recorded in the database
		  \item \textbf{Information:} Processed data in a form that is useful for decision making
		  \item \textbf{\underline{4 Levels of abstraction when discussing database:}}
					 \begin{itemize}
								\item \textbf{Real World} (For mini world or universe of discourse): Find out what 
										  actually occurs. You understand enterprise
								\item \textbf{Conceptual Model} (For mini world or universe of discourse): Start
										  figuring out what needs to be accomplished. Entities, Attributes, Relationships
								\item \textbf{Logical Model} (Occurrences): How will you represent data or how to 
										  aquire it. Record types, Data item types, Data aggregates
								\item \textbf{Data Instances} (Occurrences): Records, files, raw, tuples, database
					 \end{itemize}
		  \item \textbf{System Catalog:} ``Metadata'' about a database/table, what tables are in the table
		  \item \textbf{\underline{Data Sublanguage:}}
					 \begin{itemize}
								\item \textbf{Data Definition Language (DDL):} Describes the database
								\item \textbf{Data Manipulation Language (DML):} Process the database
								\item Data sub language commands can be embedded into host program, a general/high level
										  purpose language such as C, C++, Java, COBOL, ADA, Fortran 
										  \underline{instead of directly interacting} with SQL, etc.
					 \end{itemize}
		  \item Database should be designed for \underline{scalability} for future needs of an organization. 
					 Developor needs to develop a \underline{true conceptual model}
		  \item \textbf{\underline{(*)Steps in Staged Database Design:}}


					 (*)Know the steps and be able to explain each step of the process independently
					 \begin{enumerate}
								\item \textbf{Analyze User Environment:} Check current apps, exsisting documents,
										  interview user needs
								\item \textbf{Develop Conceptual Model:} Identify entities, attributes, relationships. 
										  Types of applications and transactions.
										  \begin{itemize}
													 \item (*)Based on an \underline{ER Diagram}.
													 \item (*)ER Diagram is composed of 
																\underline{Entity,Relationship, and Attributes}
										  \end{itemize}
								\item \textbf{Chose DBMS:} Choose an appropriate DBMS (MySQL, MS Access, Oracle, etc.)
										  \begin{itemize}
													 \item Single user VS. Multiple users
													 \item Central database VS. Distributed databases
													 \item Relational VS. Object Oriented VS. Etc. 
										  \end{itemize}
								\item \textbf{Develop Logical Model:} Map conceptual model to data model used by
										  chosen DBMS
								\item \textbf{Develop Physical Model:} Plan layout of data considering structure
										  supported by DBMS, hardware/software/resources avaliable
								\item \textbf{Evaluate Physical Model:} Develop prototype so users can be validated
										  and performaced can be measured
								\item \textbf{Tune System:} Adjust physical structures or software 
								\item \textbf{Implement System:} Use it
					 \end{enumerate}

\end{itemize}

\subsection*{Feb 6 2024}

\begin{itemize}
		  \item \textbf{Design Tools (CASE) (Computer Aided Software Engineering):} has Gantt charts, PERT charts,
					 freestanding data dictionaries.
		  \item ER Diagram is done by the \underline{Database Designer} or 
					 \underline{Database Administratior (DBA)}
		  \item \textbf{\underline{(*)Three Level Database Architecture:}}
					 \begin{itemize}
								\item Drawing in notebook
								\item Purpose of this architecture is to \underline{seperate} users model from physical
										  structure of the database
								\item \textbf{External Level (External Schema):} The way users thnk about data.
										  Ex. Name of students from Dr. Shims class, Courses Shim taught F22. 
										  (*)Many number of External schema for a single database.
								\item \textbf{Logical Level:} Middle mapping that shows independence between
										  external and physical levels. Only (*)one logical schema as its the pairing 
										  designs of the many tables. Includes 
										  \underline{entities, attributes, relationships, semantics, etc.}. Supported
										  by the internal model
								\item \textbf{Internal Level:} The way data is actually stored 
										  using \underline{data structures}
										  and \underline{file organizations}. Only (*)one internal schema. Physical
										  implementation of the database.
								\item This architecture creates good \underline{data independence} (Upper levels
										  are unaffected by changes to the lower levels
								\item \textbf{(*)Logical Data Independence:} Immunity of external models
										  to changes in the logical model
								\item \textbf{(*)Physical Data Independence:} Immunity of the logical model to changes in 
										  the internal model
					 \end{itemize}
		  \item \textbf{Entity Relationship Model:} Semantic model
		  \item \textbf{Relational Model:} Record based model
		  \item Object oriented model
		  \item Object relational model
		  \item Semi structured data model
\end{itemize}

\subsection*{Feb 8 2024}

Chapter 3

\begin{itemize}
		  \item \textbf{Entity Relationship Model:} Allows the designer to express conceptual properties
					 of the database in enterprise schema. Model describes 
					 \underline{entities, attributes, relationships.}
					 Uses its own ER Diagram.
		  \item \textbf{Entity:} Object of interest: Person, place, event, object. Can be physcial or abstract
		  \item \textbf{Entity Instance:} Represents particular object, person, class, student, etc.
					 Ex (John Doe, 35, 504-26-8585)
		  \item \textbf{Entity Type:} Representation of the data model of a category of entities. Described
					 by its name and attributes. 
		  \item \textbf{Entity Set:} Collection of entities of the same type
		  \item \textbf{Attributes:} of an entity represent defining properties of entity type
					 \begin{itemize}
								\item \textbf{Simple:} (age) or \textbf{Composite:} (1. name = first name + last name, 
										 2. address = street + city + state + zip)
							   \item \textbf{Single Valued:} (age, SSN) or \textbf{Multi Valued [use double oval]:}
										  (degree earned, color of car, email address)
								\item \textbf{Stored:} (birthday) or \textbf{Derived [use dashed oval]:}
										  (age)
					 \end{itemize}
		  \item An entity instance will have a value for each attribute. Ex (12345, Doe, John, CS, 65)
		  \item \textbf{Domain:} Set of values permiteed for each attribute. Ex age:[16,70],
					 GPA:[0,4.0], credits:[0,120]
		  \item \textbf{Null:} No value, the values of some attributes for some instances may have this value.
					 \underline{Databse should not leave attributes empty, use null}
\end{itemize}

\subsection*{X}

\begin{itemize}
		  \item \textbf{}
\end{itemize}

\end{document}

