\documentclass[12pt]{article}
\usepackage[legalpaper, margin=0.5in]{geometry}
\usepackage{soul}

\author{Tim Pasquel}

\title{CSC356 Database Systems}

\begin{document}

\maketitle

\titlepage

\section{Chapter 1:}

\subsection*{Jan 23 2024}

\begin{itemize}
		  \item \textbf{Database:} Collection of \underline{related} stored data (Ex. KU database by Microsoft Access
		  \item Having multiple copies of the same data can lead to flawed, contradictory info. 
		  \item Org. can benefit by having their data integrated into a single database.
		  \item \textbf{Integrated Database:} Collection of related data that can be used simultaneously by many departments and users in an enterprise
		  \item \textbf{Metadata:} Data about data
		  \item \textbf{``System Catalog'':} How all of the tables are related to each other/a list of them all
		  \item \textbf{Database Administrator (DBA):} Responsible for creating and maintaining the database
		  \item \textbf{Database Management Systems (DBMS):} A s/w package that allows creating and maintaining databases. All access to the database is controlled by a DBMS
		  \item Apps that are written in java, c++, etc. go through DBMS and can then present data in the 
					 form each program expects 
		  \item \textbf{End Users:} People who use the data to perform their jobs
		  \item \textbf{(*)Advantages of Integrated Database:} Sharing of data, control of redundancy, data consistency, improved data standards (Who needs to know what in each table), better data security (Ensures all data is entered correctly), improved data integrity (Same as last adv.), balancing of conflicting requirements, faster development of new apps., better data accessibility, economy of scale (Easy to modify database), more control over concurrency (More control over events that happen at the same time), better backup and recovery procedures (One repository makes less redundencies)
\end{itemize}

\subsection*{Jan 25 2024}

(*) Notes book drawings/examples

\begin{itemize}
		  \item \textbf{Relational (Tables)  Database:} They use tables which are easy to understand as well as
					 easy to implement. 
\end{itemize}

\subsection*{Jan 30 2024}

(*) Notes book drawings/examples

\textbf{Historical Developments in Databases}

\begin{itemize}
		  \item \textbf{Sequential File Processing System:} Using sequential access devices such as: punched cards in the 1890 US census, punched paper tape introduced in 1940s, magnetic tape introduced about 1950-used in UNIVAC I
		  \item \textbf{Early Database Model:} S.F.P.S (1950s): Early DB MOdels (1960s)
					 \begin{itemize}
								\item Magnetic disk introduced 1950s which was a direct access device
								\item Languages like COBOL, PL/1 for commercial processing in 1960s
								\item \underline{Hierarchical Model:} Database developed 1960s 
								\item \underline{Network Model:} Database developed 1960s
								\item both were complex, required users to understand database structures and 
										  access paths to data (Where the folders were)
					 \end{itemize}
		  \item \textbf{Relational Model:} Proposed by E.F Codd 1970 
					 \begin{itemize}
								\item Strong mathematical theory/foundation/fuctions
								\item Table model
								\item Still popularly used
								\item Similar to Microsoft Excel (?)
					 \end{itemize}
		  \item \textbf{Semantic Models:}
					 \begin{itemize}
								\item Entity relationship model (E.R Model) - P.P Chen 1976. \underline{(*)Used today}
								\item Object oriented model (O.O Model) - 1990s 
								\item Object relational mode: Object oriented capabilities in a relational database
					 \end{itemize}
		  \item \textbf{Data Warehouse:}
					 \begin{itemize}
								\item 1990s, lots of data that can be used to do statistics or, predictions to help business. 
								\item Built to store large quantities of historical data
								\item Enables fast complex quries across all of the data
								\item Typically using OLAP (Online \underline{Analytical} Processing)
								\item \underline{Note:} DB is typically for OLTP (Online Transaction Processing)
					 \end{itemize}
		  \item \textbf{Internet Access:} to a network of databases (Web based data base, publicaly avaliable)
					 \begin{itemize}
								\item Ecommerce
								\item XML standard for data exchange
					 \end{itemize}
		  \item \textbf{Big Data:} The capturing, organizing, and analyzing of massive amounts of data
					 generated by large variety of sources
\end{itemize}

\section{Chapter 2}

\subsection*{Feb 1 2024}

(*) Notes book drawings/examples

\begin{itemize}
		  \item A resource is any asset of value to an organization, incurs cost to organizations
					 operational data
		  \item Recognition of data as \underline{corperate resource} is important objective in developing
					 integrated database environment
		  \item Database secures data with a \textbf{DBMS} which provides \underline{security, integrity,
					 and realiability}
		  \item \textbf{Data:} Bare facts recorded in the database
		  \item \textbf{Information:} Processed data in a form that is useful for decision making
		  \item \textbf{\underline{4 Levels of abstraction when discussing database:}}
					 \begin{itemize}
								\item \textbf{Real World} (For mini world or universe of discourse): Find out what 
										  actually occurs. You understand enterprise
								\item \textbf{Conceptual Model} (For mini world or universe of discourse): Start
										  figuring out what needs to be accomplished. Entities, Attributes, Relationships
								\item \textbf{Logical Model} (Occurrences): How will you represent data or how to 
										  aquire it. Record types, Data item types, Data aggregates
								\item \textbf{Data Instances} (Occurrences): Records, files, raw, tuples, database
					 \end{itemize}
		  \item \textbf{System Catalog:} ``Metadata'' about a database/table, what tables are in the table
		  \item \textbf{\underline{Data Sublanguage:}}
					 \begin{itemize}
								\item \textbf{Data Definition Language (DDL):} Describes the database
								\item \textbf{Data Manipulation Language (DML):} Process the database
								\item Data sub language commands can be embedded into host program, a general/high level
										  purpose language such as C, C++, Java, COBOL, ADA, Fortran 
										  \underline{instead of directly interacting} with SQL, etc.
					 \end{itemize}
		  \item Database should be designed for \underline{scalability} for future needs of an organization. 
					 Developor needs to develop a \underline{true conceptual model}
		  \item \textbf{\underline{(*)Steps in Staged Database Design:}}


					 (*)Know the steps and be able to explain each step of the process independently
					 \begin{enumerate}
								\item \textbf{Analyze User Environment:} Check current apps, exsisting documents,
										  interview user needs
								\item \textbf{Develop Conceptual Model:} Identify entities, attributes, relationships. 
										  Types of applications and transactions.
										  \begin{itemize}
													 \item (*)Based on an \underline{ER Diagram}.
													 \item (*)ER Diagram is composed of 
																\underline{Entity,Relationship, and Attributes}
										  \end{itemize}
								\item \textbf{Chose DBMS:} Choose an appropriate DBMS (MySQL, MS Access, Oracle, etc.)
										  \begin{itemize}
													 \item Single user VS. Multiple users
													 \item Central database VS. Distributed databases
													 \item Relational VS. Object Oriented VS. Etc. 
										  \end{itemize}
								\item \textbf{Develop Logical Model:} Map conceptual model to data model used by
										  chosen DBMS
								\item \textbf{Develop Physical Model:} Plan layout of data considering structure
										  supported by DBMS, hardware/software/resources avaliable
								\item \textbf{Evaluate Physical Model:} Develop prototype so users can be validated
										  and performaced can be measured
								\item \textbf{Tune System:} Adjust physical structures or software 
								\item \textbf{Implement System:} Use it
					 \end{enumerate}

\end{itemize}

\subsection*{Feb 6 2024}

\begin{itemize}
		  \item \textbf{Design Tools (CASE) (Computer Aided Software Engineering):} has Gantt charts, PERT charts,
					 freestanding data dictionaries.
		  \item ER Diagram is done by the \underline{Database Designer} or 
					 \underline{Database Administratior (DBA)}
		  \item \textbf{\underline{(*)Three Level Database Architecture:}}
					 \begin{itemize}
								\item Drawing in notebook
								\item Purpose of this architecture is to \underline{seperate} users model from physical
										  structure of the database
								\item \textbf{External Level (External Schema):} The way users thnk about data.
										  Ex. Name of students from Dr. Shims class, Courses Shim taught F22. 
										  (*)Many number of External schema for a single database.
								\item \textbf{Logical Level:} Middle mapping that shows independence between
										  external and physical levels. Only (*)one logical schema as its the pairing 
										  designs of the many tables. Includes 
										  \underline{entities, attributes, relationships, semantics, etc.}. Supported
										  by the internal model
								\item \textbf{Internal Level:} The way data is actually stored 
										  using \underline{data structures}
										  and \underline{file organizations}. Only (*)one internal schema. Physical
										  implementation of the database.
								\item This architecture creates good \underline{data independence} (Upper levels
										  are unaffected by changes to the lower levels
								\item \textbf{(*)Logical Data Independence:} Immunity of external models
										  to changes in the logical model
								\item \textbf{(*)Physical Data Independence:} Immunity of the logical model to changes in 
										  the internal model
					 \end{itemize}
		  \item \textbf{Entity Relationship Model:} Semantic model
		  \item \textbf{Relational Model:} Record based model
		  \item Object oriented model
		  \item Object relational model
		  \item Semi structured data model
\end{itemize}

\subsection*{Feb 8 2024}

Chapter 3

\begin{itemize}
		  \item \textbf{Entity Relationship Model:} Allows the designer to express conceptual properties
					 of the database in enterprise schema. Model describes 
					 \underline{entities, attributes, relationships.}
					 Uses its own ER Diagram.
		  \item \textbf{Entity:} Object of interest: Person, place, event, object. Can be physcial or abstract
		  \item \textbf{Entity Instance:} Represents particular object, person, class, student, etc.
					 Ex (John Doe, 35, 504-26-8585)
		  \item \textbf{Entity Type:} Representation of the data model of a category of entities. Described
					 by its name and attributes. 
		  \item \textbf{Entity Set:} Collection of entities of the same type
		  \item \textbf{Attributes:} of an entity represent defining properties of entity type
					 \begin{itemize}
								\item \textbf{Simple:} (age) or \textbf{Composite:} (1. name = first name + last name, 
										 2. address = street + city + state + zip)
							   \item \textbf{Single Valued:} (age, SSN) or \textbf{Multi Valued [use double oval]:}
										  (degree earned, color of car, email address)
								\item \textbf{Stored:} (birthday) or \textbf{Derived [use dashed oval]:}
										  (age)
					 \end{itemize}
		  \item An entity instance will have a value for each attribute. Ex (12345, Doe, John, CS, 65)
		  \item \textbf{Domain:} Set of values permiteed for each attribute. Ex age:[16,70],
					 GPA:[0,4.0], credits:[0,120]
		  \item \textbf{Null:} No value, the values of some attributes for some instances may have this value.
					 \underline{Databse should not leave attributes empty, use null}. DOES NOT MEAN 0, just 
					 place holder
\end{itemize}

\subsection*{Feb 15 2024}

\begin{itemize}
		  \item \textbf{Key:} minimal subset (\underline{no proper subset of the key attributes
					 is a unique identifier}) of attributes that uniquely identify an entity. Ex.
					StuID is a key for the Student entity
		 \item \textbf{Candidate Key:} The several keys an entity can have. Keys potenialy used as the key
		 \item \textbf{Primary Key:} One key selected to uniquely identify entities. Ex. Car VIN
		 \item \textbf{Alternate Keys:} All of the other keys for an entity. Ex. Car model, year, make, etc. 
		 \item \textbf{Super Key:} A non minimal key. Ex. any key that has the primary key (unique
					identifier key), plus other identifiers. \underline{Default superkey:} the set of
					 all attributes
		 \item \textbf{Composite Key:} A key consisting of more than one attribute
		 \item \underline{(*)No primary key value can be null}
		 \item \textbf{Foreign Key:} Primary key of another relation, pointer to another relation.
					A key that can be used in multiple tables. Ex. \underline{Inventory table} has column: Vender-Code,
					 \underline{Vendor table} has column Venter-Code as well.
		 \item A table is a set of rows/specific objects/tuples
		 \item \textbf{Relationships:} Connections or interactions among entity instances
		 \item \textbf{Relationship Type:} The common properties of certain relationships 
		 \item \textbf{Relationship Set:} The cellection of relationships of that type
		 \item \textbf{(*)Degree:} The number of participating entity types (Unary - 1, Binary - 2, Tirnary - 3).
					(*)Generally keep to \underline{2}. Need to specify:
					 \begin{itemize}
								\item Cardinality
								\item Connectivity
								\item Mandatory/Optional
					 \end{itemize}
		  \item \textbf{Relationship set of degree n:} A subset of an n-ary relation of the form
		  \item \textbf{(*)Cardinality:} The number of entities to which another entity can map under
					 that relationship (min, max)
		  \item \textbf{Connectivity:} One to one (1:1), one to many (1:N), many to many (N:N)
\end{itemize}

\subsection*{Feb 20 2024}

\begin{itemize}
			\item \textbf{(*)Participation Constraints (Uese the min values of cardinarlity):}
					\begin{itemize}
							\item \underline{Total Participation:} Every memeber of entity set must participate
							in relationship, uses double
							\item \underline{Partial Participation:} Some members of entity set may not participate
							in the relationship, uses single line
					\end{itemize}
			\item \textbf{Roles:} indicate recursive relationship, the same 2 entities are realted in 
						2 different ways
			\item When data is needed to be stored, we would not be interested in data unless we had a 
			related entity already in the database. Ex. dont need sales order data unless we had 
			customer data. 
			\item \textbf{Strong Entities:} Entity Y can not exist without some X entity. X is a strong entity
			\item \textbf{Weak Entities:} In the example above, Y is the weak entity. Y must then also
			have total participation in its relationaship set with X
			\item \textbf{Identifying Relationship:} The relationship type that relates a weak entity type to its 
			owner.
			\item \textbf{Partial Key (Discriminator):} Weak entity type normally has this, 
			the set of attributes that can 
			uniquely identify weak entities that are related to the same owner entity
\end{itemize}

\subsection*{Feb 22 2024}

Chapter 4

\begin{itemize}
		\item Relational model removes details from applications providing \underline{a logical view}
			and true \underline{data independence}
		\item Relational model is based on the \underline{mathmatical notion} of a relation.
		\item Relational model is simple making it \underline{easy to underline} at a intuitive level. Allows
			speration of \underline{conceptual and physical level}. It is not nessesary to be familiar with 
			storage stuctures. \underline{Most popular model}.
		\item \textbf{Relation:} Physically represented as a \underline{table or 2D array}
		\item \textbf{Domain:} Of an attribute is the set of allowable values for the attribute
		\item \textbf{Row:} Each row of the table is an individual \underline{record or entity instance}. Also 
			called a \underline{tuple}
		\item \textbf{Columns:} Attributes of a table
		\item \underline{(*)Table:}
			\begin{itemize}
				\item Each cell has only one value 
				\item Each column has a distinct name
				\item Values in column all come from same domain
				\item Each tuple/row is distinct
				\item Order of tuples or row is immaterial
				\item When table structure is chosen, order of elements in rows must match order of column names
			\end{itemize}
\end{itemize}

\subsection*{Feb 27 2024}

\begin{itemize}
		\item \textbf{Number of Columns:} in a table is the \underline{degree} of relationship
		\item \textbf{Number of Rows:} in a table is the \underline{cardinality} of relationship. (*) This 
			term is different from the E.R diagram term
		\item Earlier systems (hierarchical and network model) were based on 
			\underline{file systems} and did not provide \underline{data independence}
		\item The \underline{relational model} provided a logical view and \underline{true data independence}
		\item The relational model is \underline{simple and easy to understand} at the intuitive level.
		\item Relational model allows separation of the \underline{conceptual and physical levels}
		\item Realtional model data operations are also easy to express making it so you do not need to know
			data structures or file systems.
		\item \textbf{(*)Integrity Constraints (IC):} rules or restrictions that apply to all instances
			of the database. \underline{Database management system} enforces these rules to ensure each 
			entry is a legal instance
		\item \textbf{(*)Key constraints:} Minimal subset of fields is a unique identifier for the tuple.
			(*)Every table that you create should have a key, no table can exsist without a key.
		\item \textbf{(*)Domain constraints:} Value of each attribute, A must be an atomic value from 
			the domain, dom(A). Ex char(9), INT, Decimal(3)
		\item \textbf{(*)Entity Integrity Constraints:} No primary key value can be null.
		\item \textbf{(*)Referentail Integrity Constraints:} If a foreign key exsists in a realtionship, then either 
			the foreign key value must match the primary key value of some tuple in its home relationship or 
			foreign key must be null
		\item Relation schema gives the table name for each relation and column headings for each of its 
			attributes. Logical model schema is the set of all these schemas for database
		\item \textbf{Relational Algebra:} Example of procedural (prescriptive) language, user tells the 
			system exactly how to manipulate the data
		\item \textbf{Relational Calculus/SQL (Structured Query Language):} Nonprocedural languages,
			the user states what data is needed but not exactly how it is to be located
		\item Both relational algebra and relational calculus are formal non user friendly languages. Not 
			implemented in their native form in database mang. systems. Both have been used as the basis
			for higher level data manipulation languages such as \underline{SQL}
\end{itemize}

\subsection*{Feb 29 2024}

\begin{itemize}
		\item A table must have some kind of key
		\item \underline{Mapping ER Model to Relational Schema (Examples in the Notes)}
			\begin{enumerate}
				\item \textbf{Strong Entity Sets:} become relations represented by base tables. They are \underline{
					non composite, single valued attributes} and become attributes of the 
					relations (column headings of table)
		  		\item ER attibutes that are \underline{composites}, make a column for each of the single attributes
					that form the compsite, or leave the composite as a single attribute.
				\item For \underline{(*)multivalued attributes}, remove them from the table, create a 
					seperate realtion
					and put the primary key of the entity with the multivalued attribute. \underline{(*)Decompose} the 
					table into pieces and multiple tables. Ex. Student(\underline{stuID}, name, major, credits) 
					and stuMajors(\underline{stuID}, \underline{major}). You can also make additional columns for
					multiple value attributes. Ex. Studnet(\underline{stuID}, name, major1, major2 
					[If none, null, not good to do], major3, credits)
		  		\item \textbf{Weak Entity Sets:} use combination of owners primary key and the discriminat as the
					key. Ex. Evaluation(\underline{facID (from the strong entity)},\underline{date (weak)},
					\underline{rater (weak)}, rating)
		  		\item \textbf{Relationship Set:} represented in tables by foreign keys or sperate relationship table,
					depending on cardinality and degree of relationship
					\begin{itemize}
						\item If A and B are strong entities, and binary relationship. \underline{A:B is one to many}.
							Place the key of A (1 side) in the table for B (many side), where it becomes a
							foreign key. (*) The opposite way doesnt work. 
						\item \underline{One to One}, put the key of either relation in the other table to show 
							connection. Put key of A in B table or vise versa, NOT BOTH. 
						\item \textbf{Binary Relationship:} that is \underline{many to many}, connection must be 
						show by a separate relationship table. The table must contain the primary key of the entities
					\end{itemize}
				\item \textbf{Ternary or n-ary Relationship:} construct table for relationship for the primary keys of
					the related entities 
				\item \textbf{Recursive Relationship:} depends on cardinarlity. If many to many, create relationship
					table. If 1 to 1 or 1 to many, the foreign key mechanism is used. 
		  	\end{enumerate}
\end{itemize}

\subsection*{Mar 21 2024}

Chapter 5 SQL

\begin{itemize}
		\item \textbf{Relational Model:} proposed by E.F Cod in 1970
		\item \textbf{SQL:} Language starting in 1974
		\item \textbf{Oracle:} Commerical relational database management system 1970
		\item \textbf{IBM's SQL/DS:} 1st relational database 1981
		\item \underline{Many} versions of SQL, also the baseline for many stystems and continiously used, some
			of the syntax may be different, but the overall is the same(*)
		\item dont worry about exact companies and dates
		\item \textbf{CREATE TABLE:} Command to create an empty table
		\item \textbf{CREATE INDEX:} Command that declares the number of indexes for the table
		\item \textbf{Index:} Used to speed up retrieval of records based on the value in one of more columns.
			Most DBMS use \underline{B trees} or \underline{B+ trees} (Data structures) for indexes
		\item Indexes speed up your search so that you do not retrieve information that you do not need
		\item (*) Explain the concept of indexes as the data structures associated with them, what the adv. are.
			You do \underline{not} need an index for every attributes because that would waste space
		\item At the physical level, base tables and indexes are represented in files
		\item User is unaware that indexes exsist, have no control over which index will be used in locating a
			record
		\item \textbf{CREATE VIEW:} Command to view the desired table. Creates a ``Virtual table'' and will 
			\underline{not} be permanently stored
		\item \textbf{Dynamic database definitions:} Feature of a relational database, can create new tables, 
			add columns to old ones, create new indexes, define views, drop objects at any time
		\item (*) Foreign key is a primary key of another table
		\item \textbf{Examples:}
			\begin{itemize}
				\item CREATE TABLE [schema-name] base-table-name (colname datatype [column constraints]
						[, colname datatype [column constraints]] ... [table constraints] [storage specifications]);
				\item CREATE TABLE Customer (cno CHAR(3), balance NUMBER(5));
				\item CREATE TABLE Employee(

						SSN CHAR(9) NOT NULL,				CHAR(n) - fixed length string of length n

						Name VARCHAR2(30) NOT NULL, 		VARCHAR2(n) - variable length string of max length n

						AGE INT, DEFAULT 0

						PRIMARY KEY(SSN)

						);
			  \item CREATE TABLE Works\_On(

						ESSN CHAR(9) NOT NULL,

						PNO INT NOT NULL,

						HOURS DECIMAL(3,1) NOT NULL,      (3,1) - 3 digits, 1 after decimal Ex. 99.1

						PRIMARYKEY(ESSN,PNO),

						FOREIGN KEY(ESSN) REFERENCES EMPLOYEE(SSN),
						
						FOREIGN KEY(PNO) REFERENCES PROJECT(PNUMBER)

						);			
			\end{itemize}
\end{itemize}

\subsection*{Mar 26 2024}

\begin{itemize}
		\item Base table use user supplied name. \underline{No SQL Key words} can be used as the name, name
			must be unique for the table
		\item \textbf{Each Column:} Specific name unique in table, a data type
		\item SQL statements are case insensitive, everything will be displayed in uppercase
		\item (*) [Test will write SQL statements] Each line ends with a comma, except for the last, 
			which ends in a semi colon
		\item \textbf{Data Types:} Vary from DBMS to DBMS
		\item \textbf{VARCHAR2 (n):} stores varying length strings with max size of n bytes. N to 4000 bytes
		\item \textbf{CHAR (n):} stores fixed length strings, max size of 2000 bytes
		\item \textbf{Fixed Point Numbers:} NUMBER (p, s). p - total number of digits, s - number of digits to
	  		the right of the decimal point. For ints, s is omitted. Ex. NUMBER (6,2): 9999.99,
			NUMBER(6): 999999	
		\item \textbf{DATE:} default format ``dd-mon-yy'' (Ex. ``02-DEC-11''). (*)Month is 3 letters 
		\item \textbf{FLOAT (p):} p is the number of bytes  
		\item \textbf{Integrity Constraints:} DBMS enforces data correctness, allows only legal instances to be
	  		created	
		\item \textbf{CREATE TABLE command:} has optional constraints, column level in line constraints,
	  		table level out of line constraints	
		\item \textbf{Column Level constraints:} Sets column constraints such as NOT NULL, NULL, PRIMARY KEY,
			FOREIGN KEY, REF, CHECK, DEFAULT
		\item \textbf{Table Level constraints:} Sets table constraints such as primary key, foreign keys 
		\item If primary key is not composite, possible to specify PRIMARY KEY as column constraint, add
	  		PRIMARY KEY after the data type column. Ex. VARCHAR2 (6) PRIMARY KEY,	
		\item \textbf{CHECK:} specifys conditions that the rows of the table cant violate, verifys that
  			values provided for attributes are appropriate		
		\item \textbf{DEFAULT:} specify defulat value for column, Ex. credits NUMBER(6) DEFAULT 0, 
		\item We can optionally provide a name for a constraint, otherwise system wil provide a name. User defined
	  		name is prefered	
		\item Table constraints appear \underline{after} all the columns have been declared
		\item If primary key is \underline{composite}, must be identified as a table constraint
		\item \textbf{FOREIGN KEY:} Constraint requires that we identify the referenced table where the column
			or column combination appears
		\item \textbf{CONSTRAINT:} (*)must use this keywork for out of line constraint (Table level constraints), 
			can be followed by an identifer. Ex. CONSTRAINT Class\_facID\_fk FOREIGN KEY(facId) REFERENCES (facdId)
			ON DELETE CASCADE  
		\item \textbf{ON DELETE CASCADE:} delete all class records for that faculty member
		\item \textbf{ON DELETE SET NULL:} set the facId in the class value to null value
		\item \textbf{UNIQUE:} can be used to specify that values in a combination of columns must be unique 
		\item \textbf{} 
\end{itemize}

\subsection*{Feb 22 2024}

\begin{itemize}
		\item
\end{itemize}

\subsection*{Feb 22 2024}

\begin{itemize}
		\item
\end{itemize}

\subsection*{Feb 22 2024}

\begin{itemize}
		\item
\end{itemize}

\subsection*{Feb 22 2024}

\begin{itemize}
		\item
\end{itemize}



\end{document}

