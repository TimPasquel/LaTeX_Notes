\documentclass[12pt]{article}
\usepackage[legalpaper, margin=0.5in]{geometry}
\usepackage{soul}

\author{Tim Pasquel}

\title{CSC356 Database Systems}

\begin{document}

\maketitle

\titlepage

\section{Chapter 1:}

\subsection{Jan 23 2024}

\begin{itemize}
		  \item \textbf{Database:} Collection of \underline{related} stored data (Ex. KU database by Microsoft Access
		  \item Having multiple copies of the same data can lead to flawed, contradictory info. 
		  \item Org. can benefit by having their data integrated into a single database.
		  \item \textbf{Integrated Database:} Collection of related data that can be used simultaneously by many departments and users in an enterprise
		  \item \textbf{Metadata:} Data about data
		  \item \textbf{``System Catalog'':} How all of the tables are related to each other/a list of them all
		  \item \textbf{Database Administrator (DBA):} Responsible for creating and maintaining the database
		  \item \textbf{Database Management Systems (DBMS):} A s/w package that allows creating and maintaining databases. All access to the database is controlled by a DBMS
		  \item Apps that are written in java, c++, etc. go through DBMS and can then present data in the 
					 form each program expects 
		  \item \textbf{End Users:} People who use the data to perform their jobs
		  \item \textbf{(*)Advantages of Integrated Database:} Sharing of data, control of redundancy, data consistency, improved data standards (Who needs to know what in each table), better data security (Ensures all data is entered correctly), improved data integrity (Same as last adv.), balancing of conflicting requirements, faster development of new apps., better data accessibility, economy of scale (Easy to modify database), more control over concurrency (More control over events that happen at the same time), better backup and recovery procedures (One repository makes less redundencies)
\end{itemize}

\subsection{Jan 25 2024}

(*) Notes book drawings/examples

\begin{itemize}
		  \item \textbf{Relational (Tables)  Database:} They use tables which are easy to understand as well as
					 easy to implement. 
\end{itemize}

\subsection{Jan 30 2024}

(*) Notes book drawings/examples

\textbf{Historical Developments in Databases}

\begin{itemize}
		  \item \textbf{Sequential File Processing System:} Using sequential access devices such as: punched cards in the 1890 US census, punched paper tape introduced in 1940s, magnetic tape introduced about 1950-used in UNIVAC I
		  \item \textbf{Early Database Model:} S.F.P.S (1950s): Early DB MOdels (1960s)
					 \begin{itemize}
								\item Magnetic disk introduced 1950s which was a direct access device
								\item Languages like COBOL, PL/1 for commercial processing in 1960s
								\item \underline{Hierarchical Model:} Database developed 1960s 
								\item \underline{Network Model:} Database developed 1960s
								\item both were complex, required users to understand database structures and 
										  access paths to data (Where the folders were)
					 \end{itemize}
		  \item \textbf{Relational Model:} Proposed by E.F Codd 1970 
					 \begin{itemize}
								\item Strong mathematical theory/foundation/fuctions
								\item Table model
								\item Still popularly used
								\item Similar to Microsoft Excel (?)
					 \end{itemize}
		  \item \textbf{Semantic Models:}
					 \begin{itemize}
								\item Entity relationship model (E.R Model) - P.P Chen 1976. \underline{(*)Used today}
								\item Object oriented model (O.O Model) - 1990s 
								\item Object relational mode: Object oriented capabilities in a relational database
					 \end{itemize}
		  \item \textbf{Data Warehouse:}
					 \begin{itemize}
								\item 1990s, lots of data that can be used to do statistics or, predictions to help business. 
								\item Built to store large quantities of historical data
								\item Enables fast complex quries across all of the data
								\item Typically using OLAP (Online \underline{Analytical} Processing)
								\item \underline{Note:} DB is typically for OLTP (Online Transaction Processing)
					 \end{itemize}
		  \item \textbf{Internet Access:} to a network of databases (Web based data base, publicaly avaliable)
					 \begin{itemize}
								\item Ecommerce
								\item XML standard for data exchange
					 \end{itemize}
		  \item \textbf{Big Data:} The capturing, organizing, and analyzing of massive amounts of data
					 generated by large variety of sources
\end{itemize}

\subsection{X}

\begin{itemize}
		  \item \textbf{}
\end{itemize}


\end{document}

